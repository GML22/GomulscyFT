\chapter{Gomulscy z~Desna}

% Przednia okładka podrozdziału
\includepdf{Desno_mapa_fin.png}

\lettrine[lines=5, lhang=0.1]{D}{esno} jako obszar, jeszcze nie samodzielna
wieś, po raz pierwszy zostało wspomniane w~\enquote{Inwentarzu dóbr dzierżawy
Dembe} sporządzonym w~1772~roku.

\newpage
\ifodd\value{page}\hbox{}\newpage\fi

\section*{Potomkowie Michała i~Marianny Gomulskich}

Pierwszym członkiem rodziny Gomulskich, który osiedlił się na terenie obecnego
Desna był Michał Gomulski (ur. 1859~r. - zm. 1918~r.), przedostatni syn Piotra
Franciszka Gomulskiego i~Marianny z~domu Chłopik, młodszy brat Jana
Gomulskiego, którego losy były omawiane w~poprzednim podrozdziale ninijszej
książki. Michał Gomulski 27~lutego 1881~roku ożenił się z~Marianną Wróbel
(ur. 1860~r. - zm. 1934~r.) w~kościele pod wezwaniem Narodzenia Najświętszej
Maryi Panny w~Mińsku Mazowieckim (patrz ryc.\ref{fig:mgomulski_1881}).
Marianna Wróbel urodziła się w~Siennicy (patrz ryc. \ref{fig:mwrobel_1860}),
była córką Jana Wróbla, pracującego jako służący w~tamtejszym probostwie, oraz
Marianny z~domu Drozd. Marianna w~momencie ślubu z~Michałem mieszkała
w~Mikanowie, stąd też nowożeńcy przez pierwsze trzynaście lat małżeństwa
mieszkali właśnie w~tej miejscowości.

\begin{figure}[!ht]
    \vspace*{0.5cm}
    \centering \includegraphics[width=1.0\linewidth]{
        1860_Marianna_Wróbel_akt_chrztu_parafia_Siennica_wpis_90.jpg}
    \captionsetup{format=hang}
    \caption{Akt chrztu Marianny Wróbel - par. Siennica 1860~rok (90/1860) 
    \cite{par_siennica}.}
    \label{fig:mwrobel_1860}
\end{figure}

Michał i~Marianna Gomulscy dochowali się jedenaściorga dzieci, z~czego
pięcioro dożyło wieku dorosłego\footnote{Drzewo genealogiczne najbliższej
rodziny Michała i~Marianny Gomulskich zostało zaprezentowane
w~\hyperref[sec:drzewo_mgomulski]{załączniku numer I}~do niniejszej książki.}:

\begin{itemize}
    \item Antoni Gomulski (ur. 1882~r. - zm. 1926~r.) - patrz: ryc. 
    \ref{fig:agomulski_1882},
    \item Stanisław Gomulski (ur. 1883~r. - zm. 1884~r.) - patrz: ryc. 
    \ref{fig:sgomulski_1883},
    \item Marcin Gomulski (ur. 1885~r. - zm. ?) - patrz: ryc. 
    \ref{fig:mgomulski_1885},
    \item Jan Gomulski (ur. 1888~r. - zm. 1923~r.) - patrz: ryc. 
    \ref{fig:jgomulski_1888},
    \item Władysław Gomulski (ur. 1890~r. - zm. 1890~r.) - patrz: ryc. 
    \ref{fig:wgomulski_1890},
    \item Stanisław Gomulski (ur. 1892~r. - zm. 1961~r.) - patrz: ryc. 
    \ref{fig:sgomulski_1892},
    \item Zofia Gomulska (ur. 1894~r. - zm. 1895~r.) - patrz: ryc. 
    \ref{fig:zgomulska_1894},
    \item Stefan Gomulski (ur. 1896~r. - zm. 1923~r.) - patrz: ryc. 
    \ref{fig:sgomulski_1896},
    \item Józef Gomulski (ur. 1899~r. - zm. 1902~r.) - patrz: ryc. 
    \ref{fig:jgomulski_1899},
    \item Feliks Gomulski (ur. 1901~r. - zm. 1902~r.) - patrz: ryc. 
    \ref{fig:fgomulski_1901},
    \item Józefa Gomulska (ur. 1904~r. - zm. 1987~r.) - patrz: ryc. 
    \ref{fig:jgomulska_1904}.
\end{itemize}

\begin{figure}[!ht]
    \vspace*{0.5cm}
    \centering \includegraphics[width=0.88\linewidth]{
        1881_Michał_Gumulski_Marianna_Wróbel_akt_ślubu_parafia_Mińsk_Mazowiecki_wpis_26.jpg}
    \captionsetup{format=hang}
    \caption{Akt ślubu Michała Gomulskiego i~Marianny Wróbel - par. 
    Mińsk Mazowiecki 1881~rok (26/1881) \cite{par_minsk2}.}
    \label{fig:mgomulski_1881}
\end{figure}

\begin{figure}[!ht]
    \vspace*{0.5cm}
    \centering \includegraphics[width=0.77\linewidth]{
        1882_Antoni_Gomulski_akt_chrztu_parafia_Mińsk_Mazowiecki_wpis_21.jpg}
    \captionsetup{format=hang}
    \caption{Akt chrztu Antoniego Gomulskiego - par. Mińsk Mazowiecki
    1894~rok (21/1882)
    \cite{par_minsk2}.}
    \label{fig:agomulski_1882}
\end{figure}

\begin{figure}[!ht]
    \vspace*{0.5cm}
    \centering \includegraphics[width=0.77\linewidth]{
        1883_Stanisław_Gomulski_akt_chrztu_parafia_Mińsk_Mazowiecki_wpis_264.jpg}
    \captionsetup{format=hang}
    \caption{Akt chrztu Stanisława Gomulskiego - par. Mińsk Mazowiecki
    1883~rok (264/1883)
    \cite{par_minsk2}.}
    \label{fig:sgomulski_1883}
\end{figure}

\begin{figure}[!ht]
    \vspace*{0.5cm}
    \centering \includegraphics[width=0.77\linewidth]{
        1885_Marcin_Gomulski_akt_chrztu_parafia_Mińsk_Mazowiecki_wpis_300.jpg}
    \captionsetup{format=hang}
    \caption{Akt chrztu Marcina Gomulskiego - par. Mińsk Mazowiecki
    1883~rok (300/1885)
    \cite{par_minsk2}.}
    \label{fig:mgomulski_1885}
\end{figure}

\begin{figure}[!ht]
    \vspace*{0.5cm}
    \centering \includegraphics[width=0.77\linewidth]{
        1888_Jan_Gomulski_akt_chrztu_parafia_Mińsk_Mazowiecki_wpis_144.jpg}
    \captionsetup{format=hang}
    \caption{Akt chrztu Jana Gomulskiego - par. Mińsk Mazowiecki
    1888~rok (144/1888)
    \cite{par_minsk2}.}
    \label{fig:jgomulski_1888}
\end{figure}

\begin{figure}[!ht]
    \vspace*{0.5cm}
    \centering \includegraphics[width=0.72\linewidth]{
        1890_Władysław_Gomulski_akt_chrztu_parafia_Mińsk_Mazowiecki_wpis_315.jpg}
    \captionsetup{format=hang}
    \caption{Akt chrztu Władysława Gomulskiego - par. Mińsk Mazowiecki
    1890~rok (315/1890)
    \cite{par_minsk2}.}
    \label{fig:wgomulski_1890}
\end{figure}

\begin{figure}[!ht]
    \vspace*{0.5cm}
    \centering \includegraphics[width=0.72\linewidth]{
        1892_Stanisław_Gomulski_akt_chrztu_parafia_Mińsk_Mazowiecki_wpis_47.jpg}
    \captionsetup{format=hang}
    \caption{Akt chrztu Stanisława Gomulskiego - par. Mińsk Mazowiecki
    1892~rok (47/1892)
    \cite{par_minsk2}.}
    \label{fig:sgomulski_1892}
\end{figure}

\begin{figure}[!ht]
    \vspace*{0.5cm}
    \centering \includegraphics[width=0.77\linewidth]{
        1894_Zofia_Gomulska_akt_chrztu_parafia_Długa_Kościelna_wpis_118.jpg}
    \captionsetup{format=hang}
    \caption{Akt chrztu Zofii Gomulskiej - par. Długa Kościelna
    1894~rok (118/1894)
    \cite{par_dluga}.}
    \label{fig:zgomulska_1894}
\end{figure}

\begin{figure}[!ht]
    \vspace*{0.5cm}
    \centering \includegraphics[width=0.77\linewidth]{
        1896_Stefan_Gomulski_akt_chrztu_parafia_Długa_Kościelna_wpis_56.jpg}
    \captionsetup{format=hang}
    \caption{Akt chrztu Stefena Gomulskiego - par. Długa Kościelna
    1896~rok (56/1896)
    \cite{par_dluga}.}
    \label{fig:sgomulski_1896}
\end{figure}

\begin{figure}[!ht]
    \vspace*{0.5cm}
    \centering \includegraphics[width=0.82\linewidth]{
        1899_Józef_Gomulski_akt_chrztu_parafia_Długa_Kościelna_wpis_36.jpg}
    \captionsetup{format=hang}
    \caption{Akt chrztu Józefa Gomulskiego - par. Długa Kościelna
    1899~rok (36/1899)
    \cite{par_dluga}.}
    \label{fig:jgomulski_1899}
\end{figure}

\begin{figure}[!ht]
    \vspace*{0.5cm}
    \centering \includegraphics[width=0.82\linewidth]{
        1901_Feliks_Gomulski_akt_chrztu_parafia_Długa_Kościelna_wpis_124.jpg}
    \captionsetup{format=hang}
    \caption{Akt chrztu Feliksa Gomulskiego - par. Długa Kościelna
    1901~rok (124/1901)
    \cite{par_dluga}.}
    \label{fig:fgomulski_1901}
\end{figure}

\begin{figure}[!ht]
    \vspace*{0.5cm}
    \centering \includegraphics[width=0.82\linewidth]{
        1904_Józefa_Gomulska_Ładno_akt_chrztu_parafia_Długa_Kościelna_wpis_46.jpg}
    \captionsetup{format=hang}
    \caption{Akt chrztu Józefy Gomulskiej - par. Długa Kościelna
    1904~rok (46/1904)
    \cite{par_dluga}.}
    \label{fig:jgomulska_1904}
\end{figure}

Około 1894~roku Michał i~Marianna Gomulscy wraz z~rodziną przeprowadzili się
do oddalonej o~około 18 kilometrów na północny zachód wsi Cisie, gdzie
przyszła na świat ich pierwsza córka Zofia.

\newpage
\ifodd\value{page}\hbox{}\newpage\fi

\section*{Potomkowie Stanisława i~Elżbiety Gomulskich}

Drugim członkiem rodziny Gomulskich, który osiedlił się na terenie obecnego
Desna był Stanisław Gomulski (ur. 1863~r. - zm. 1929~r.), ostatni syn Piotra
Franciszka Gomulskiego i~Marianny z~domu Chłopik, najmłodszy brat Jana
Gomulskiego (ur.~1836~r. - zm.~1901~r.).

\newpage
\ifodd\value{page}\hbox{}\newpage\fi

\section*{Potomkowie Stanisława i~Julianny Gomulskich}

Ostatnim członkiem rodziny Gomulskich, który osiedlił się na terenie obecnego
Desna był Stanisław Gomulski (ur. 1861~r. - zm. 1910~r.), drugi syn Jana
Gomulskiego i~Agnieszki z~domu Piotrkowicz, wnuk Piotra Franciszka Gomulskiego
i~Marianny z~domu Chłopik.